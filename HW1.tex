\documentclass[12pt]{article}
\usepackage[utf8]{inputenc}
\usepackage[margin=1in]{geometry}
\usepackage{amsmath, amssymb, amsthm}
\usepackage{enumitem}
\usepackage[colorlinks]{hyperref}
\usepackage{graphicx}
\usepackage{tikz}
\usepackage{multicol}
\usetikzlibrary{arrows}
\usetikzlibrary{calc, positioning, fit, shapes.misc}

\title{MATH 5707 (Spring 2026): Homework 1}
\author{}
\date{}

\begin{document}

\maketitle

\section*{Directions and Introduction}

Submit a pdf of your solutions to the HW 1 assignment on Gradescope by 11:59 pm on [Date].

Problem 2 is marked as a peer review problem. \href{https://docs.google.com/document/d/1jSw9pmMJJFUx_6dTkcUi4k4HJNIrrIIgs9zdWhAycx0/edit?usp=sharing}{This document} gives directions and deadlines for the peer review process.

When working on this assignment, you should focus on the following goals:
\begin{itemize}
    \item Demonstrate fluency with the skill of proving that two graphs are or are not isomorphic.
    \item Clearly communicate solutions using complete sentences and enough explanation that another 5707 student could follow your work.
    \item Demonstrate fluency with the concept of degree sequences.
    \item Translate a scenario or problem into a question or statement about graphs.
    \item Demonstrate fluency with basic graph theory terminology and notation, such as $\nu(G)$, $\varepsilon(G)$, $\bar{G}$, bipartite, complete, simple graph, multigraph, cycles, etc.
\end{itemize}


\section*{Problems}

\begin{enumerate}[label=\arabic*.]
    \item Draw a simple graph $G$ that satisfies the following properties, or explain why no such graph can exist.
    \begin{itemize}
        \item $G$ is bipartite.
        \item $\varepsilon(\bar{G})=12$.
        \item $G$ contains a cycle of length $6$.
        \item $\nu(G)$ is odd.
    \end{itemize}

    \item(Peer Review Problem; 1.5.7(a) from Bondy-Murty)
Let $\mathbf{d}=(d_1,d_2,\dots, d_n)$ be a non-increasing sequence of non-negative integers, and denote the sequence $(d_2-1, d_3-1,\dots,d_{d_1+1}-1,d_{d_1+2},\dots, d_n)$ by $\mathbf{d}'$. Show that, $\mathbf{d}$ is graphic if and only if $\mathbf{d}'$ is graphic.


    \item For each of the following sets of properties, either give an example graph that satisfies those properties or explain why no graph can exist.
    \begin{enumerate}[label=(\alph*)]
        \item A multigraph with degree sequence $(7, 6, 5, 4, 3, 3, 2)$.
        \item A simple graph with degree sequence $(7, 6, 5, 4, 3, 3, 2)$.
    \end{enumerate}

    \item Determine which pairs of the following graphs are isomorphic. Justify your answers.
    
  \def\gap{6}
		\def\bigrad{2}
		\def\smrad{1}
		\begin{center}
			\begin{tikzpicture}[scale=.7]
				\useasboundingbox (-\bigrad,-\bigrad) rectangle (\bigrad+3*\gap,\bigrad);
					\foreach \i in {0,1,2}{
						\foreach \j in {0,1,2,3,4}{
							  \node[inner sep=1.8pt,fill,circle]    (\i;inner\j) at ($(\i*\gap,0)+(90-\j*72:\smrad)$){};
							  \node[inner sep=1.8pt,fill,circle]    (\i;outer\j) at ($(\i*\gap,0)+(90-\j*72:\bigrad)$){};
						}	
					}
					\foreach \j in {0,1,...,8}{
						  \node[inner sep=1.8pt,fill,circle]    (3;\j) at ($(3*\gap,0)+(90-\j*40:\bigrad)$) {};
					}
					  \node[inner sep=1.8pt,fill,circle]    (3;hub) at (3*\gap,0) {};
					\foreach \j in {0,1,2,3,4}{
						\pgfmathtruncatemacro{\next}{Mod(\j+1,5)};
						\pgfmathtruncatemacro{\sec}{Mod(\j+2,5)};
						\draw   (0;inner\j) -- (0;inner\next);
						\draw   (0;outer\j) -- (0;outer\next);
						\draw   (0;inner\j) -- (0;outer\j);
						\draw   (1;inner\j) -- (1;inner\sec);
						\draw   (1;outer\j) -- (1;outer\next);
						\draw   (1;inner\j) -- (1;outer\j);
						\draw   (2;inner\j) -- (2;outer\next);
						\draw   (2;outer\j) -- (2;inner\next);
						\draw   (2;inner\j) -- (2;outer\j);
					}
					\foreach \j in {0,1,...,8}{
						\pgfmathtruncatemacro{\next}{Mod(\j+1,9)};
						\draw   (3;\j) -- (3;\next);
					}
					\foreach \j in {0,3,6}{
						\draw   (3;hub) -- (3;\j);
					}
					\draw   (3;1) -- (3;5);
					\draw   (3;4) -- (3;8);
					\draw   (3;2) -- (3;7);
			\end{tikzpicture}
		\end{center}

    \item Recall the Jumping Julia puzzle game we played on the first day of class. (\href{https://docs.google.com/document/d/1j-n6fgk1Z_MamSJ3g9W1Dm7nC-Zcp1TIzYb_CxZFeyQ/edit?usp=sharing}{See activity here.})
    \begin{enumerate}[label=(\alph*)]
        \item Consider Puzzle 1. Draw a graph on 16 vertices that represents the puzzle. Clearly state the criteria you are using to determine if two vertices are adjacent.
        \item Explain how you can use the graph you drew to solve the Jumping Julia puzzle.
    \end{enumerate}

\end{enumerate}

\end{document}
